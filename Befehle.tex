% !TEX encoding = UTF-8 Unicode
% !TEX root =  Bachelorarbeit.tex

% Abkürzungen mit korrektem Leerraum --------------------------------------------------------------------
\newcommand{\ua}{\mbox{u.\,a.\ }}
\newcommand{\zB}{\mbox{z.\,B.\ }}
\newcommand{\dahe}{\mbox{d.\,h.\ }}
\newcommand{\Vgl}{Vgl.\ }
\newcommand{\bzw}{bzw.\ }
\newcommand{\evtl}{evtl.\ }
\newcommand{\etc}{etc.\ }

\newcommand{\Abbildung}[1]{Abbildung~\ref{fig:#1}}

\newcommand{\bs}{$\backslash$}


% Erzeugt ein Listenelement mit fetter Überschrift ------------------------------------------------------
\newcommand{\itemd}[2]{\item{\textbf{#1}}\\{#2}}


% Einige Befehle zum Zitieren -------------------------------------------------------------------------

% Fußnoten Zitat für Webseiten im THM-Stiel
\DeclareCiteCommand{\footWWWcite}[\mkbibfootnote]
  {\usebibmacro{prenote}}
  {\usebibmacro{citeindex}%
    \printtext{Vgl.}
    \printnames{author}
    \
    \printfield[citetitle]{title}%
    \
   	\mkbibparens{
		\printfield{year}
	}
	\printfield{url}
	\usebibmacro{urldate}
	}
  {\addsemicolon\space}
  {\usebibmacro{postnote}}
 
% Fußnoten Zitat im THM-Stiel  
\DeclareCiteCommand{\footmycite}[\mkbibfootnote]
  {\usebibmacro{prenote}}
  {\usebibmacro{citeindex}%
   \printtext{Vgl.}
   \setunit{\addnbspace}
   \printnames{labelname}%
   \setunit{\labelnamepunct}
   \printfield[citetitle]{title}%
   \ \mkbibparens{\printfield{year}}
  }
  {\addsemicolon\space}
  {\usebibmacro{postnote}}

% Zitat im THM-Stiel
\DeclareCiteCommand{\mycite}
 {\usebibmacro{prenote}}
  {\printtext{Vgl.}
  \printnames{author}
  \printfield[citetitle]{title}%
  \newunit
    \ \mkbibparens{\printfield{year}}}
  {\addsemicolon\space}
  {\usebibmacro{postnote}}

% Zitat für Webseiten im THM-Stiel
\DeclareCiteCommand{\myWWWcite}
  {\usebibmacro{prenote}}
  {\usebibmacro{citeindex}%
   \printtext{Vgl.}
   \setunit{\addnbspace}
   \printnames{labelname}%
   \setunit{\labelnamepunct}
   \printfield[citetitle]{title}%
   \ \mkbibparens{\printfield{year}}
  }
  {\addsemicolon\space}
  {\usebibmacro{postnote}}


% zum Ausgeben von Autoren
\newcommand{\AutorName}[1]{\textsc{#1}}
\newcommand{\Autor}[1]{\AutorName{\citeauthor{#1}}}

% verschiedene Befehle um Wörter semantisch auszuzeichnen -----------------------------------------------
\newcommand{\NeuerBegriff}[1]{\textbf{#1}}

\newcommand{\Fachbegriff}[2][\empty]{\ifthenelse{\equal{#1}{\empty}}{\textit{#2}\xspace}{\textit{#2}\xspace\footnote{#1}\nomenclature{#2}{#1}}}

\newcommand{\FachbegriffSpezialA}[4]{\textit{#4}\footnote{#3}\label{fn:#1}\nomenclature{#4}{#2. Siehe auch Fu{\ss}zeile auf Seite~\pageref{fn:#1}.}}

\newcommand{\FachbegriffSpezialB}[5]{\textit{#5}\footnote{#3}\label{fn:#1}\nomenclature{#4}{#2. Siehe auch Fu{\ss}zeile auf Seite~\pageref{fn:#1}.}}


% Beträge mit Währung -----------------------------------------------------------------------------------
\newcommand{\Betrag}[2][general]{#2\,\ifthenelse{\equal{#1}{dollar}}{\$}{}\ifthenelse{\equal{#1}{euro}}{€}{}\ifthenelse{\equal{#1}{yen}}{¥}{}\ifthenelse{\equal{#1}{cent}}{¢}{}\ifthenelse{\equal{#1}{pound}}{£}{}\ifthenelse{\equal{#1}{peso}}{₱}{}\ifthenelse{\equal{#1}{baht}}{฿}{}\ifthenelse{\equal{#1}{franc}}{₣}{}\ifthenelse{\equal{#1}{lira}}{₤}{}\ifthenelse{\equal{#1}{drachma}}{₯}{}\ifthenelse{\equal{#1}{pfennig}}{₰}{}\ifthenelse{\equal{#1}{general}}{¤}{}}


% Sonstiges ---------------------------------------------------------------------------------------------
% Fußnoten Layout für den Text Teil
\newcommand{\resetFootHead}[1]{
\setcounter{page}{#1}
\ihead{\small{\textit{\autor}}
       \\
       \textsc{\chapterName}
       \\[1ex]
       \headmark
    }
\chead{}
\ohead{\includegraphics[height=1.5cm]{\logoSP}}
\ifoot{}
\cfoot{}
\ofoot{\pagemark}
}
% Fußnoten Layout für den Verzeichnis-Teil
\newcommand{\verFootHead}[1]{
\setcounter{page}{#1}
\ihead{\textit{\headmark}}
\chead{}
\ohead{Verzeichnisse-\pagemark}
\ifoot{\autor}
\cfoot{}
\ofoot{\includegraphics[height=1.2cm]{\logoSP}}
}

\newcommand{\setChapName}[1]{\renewcommand{\chapterName}{#1}}
\newcommand{\qMarks}[1]{\openautoquote#1\closeautoquote}
\newcommand\tab[1][1cm]{\hspace*{#1}}


\newcommand{\lref}[1]{\lstlistingname \  \ref{#1} auf der Seite \pageref{#1}}
\newcommand{\laref}[1]{\lstlistingname \  \ref{#1} im Anhang auf der Seite \pageref{#1}}

\newcommand{\figRef}[1]{\figurename \ \ref{#1}}
\newcommand{\tabRef}[1]{\tablename \ \ref{#1}}

\newcommand{\myHuge}[1]{\huge {#1} \\ \normalsize}
\newcommand{\myLarge}[1]{\large {#1} \\ \normalsize}


% A footnote without marker in the text
\newcommand\plainfootnote[1]{%
  \begingroup
  \renewcommand\thefootnote{}\footnote{#1}%
  \addtocounter{footnote}{-1}%
  \endgroup
}