% !TEX encoding = UTF-8 Unicode
% !TEX root =  Bachelorarbeit.tex

% Meta-Informationen ------------------------------------------------------------------------------------
%   Definition von globalen Parametern, die im gesamten Dokument verwendet
%   werden können (z.B auf dem Deckblatt etc.).
%
%   ACHTUNG: Wenn die Texte Umlaute oder ein Esszet enthalten, muss der folgende
%            Befehl bereits an dieser Stelle aktiviert werden:
%            \usepackage[latin1]{inputenc}
% -------------------------------------------------------------------------------------------------------
\newcommand{\titel}{Titel}
\newcommand{\untertitel}{Untertitel}
\newcommand{\autor}{Max Mustermann}
\newcommand{\adrStrasse}{Strasse N}
\newcommand{\adrOrt}{PLZ Ort}

\newcommand{\matrikelnr}{1337}
\newcommand{\hochschulBetreuer}{Prof. Dr. Mustermann}
\newcommand{\fachbetreuer}{Frau Mustermann}
\newcommand{\unternehmen}{Pseudofirma GmbH}

\newcommand{\jahr}{2020}
\newcommand{\keywords}{Bachelor of Science (B.Sc.),}
\newcommand{\studienbereich}{Studiengang\xspace}
\newcommand{\hochschule}{Technischen Hochschule Mittelhessen}
\newcommand{\ort}{Bad Vilbel}
\newcommand{\logoTHM}{THMLogo.jpeg}
\newcommand{\logoSP}{StudiumPlusLogo.jpeg}

\newcommand{\chapterName}{ChapterName}
