% !TEX encoding = UTF-8 Unicode
% !TEX root =  Bachelorarbeit.tex

% Grafiken -----------------------------------------------------------------------------
% Einbinden von JPG-Grafiken ermöglichen
\usepackage[dvips,final]{graphicx}

% Rotation von Bildern ermöglichen
\usepackage{rotating}

% hier liegen die Bilder des Dokuments
\graphicspath{{Bilder/}}

% Anpassung des Seitenlayouts ---------------------------------------------------
%   siehe Seitenstil.tex
% -----------------------------------------------------------------------------------------
\usepackage[
    automark, % Kapitelangaben in Kopfzeile automatisch erstellen
    headsepline, % Trennlinie unter Kopfzeile
    ilines % Trennlinie linksbündig ausrichten
]{scrlayer-scrpage}

\automark[section]{section}

% Anpassung an Landessprache -------------------------------------------------
\usepackage[ngerman]{babel}

\usepackage{textcomp} % Euro-Zeichen etc.


% Schrift --------------------------------------------------------------------------------
\usepackage{lmodern} % bessere Fonts

% Arial
\usepackage{helvet}
\renewcommand{\familydefault}{\sfdefault}

% Wird für fette Labels im Literaturverzeichnis benötigt
\usepackage{xpatch}

\usepackage[document]{ragged2e}

% URL verlinken, lange URLs umbrechen etc. -------------------------------------
\usepackage{url}

% biblatex einbinden -----------------------------------------------------------------------
\usepackage[backend=biber, hyperref=true, backref=true, backrefstyle=all+, bibstyle=authoryear, sorting=nyt, dashed=false, urldate=long, natbib=true, citestyle=authoryear-ibid]{biblatex}

%\usepackage[square,sort,comma,numbers]{natbib}
%\usepackage{cite}


\usepackage[babel,german=quotes]{csquotes}

% zum Einbinden von Programmcode -----------------------------------------------
% um Code einzubinden
\usepackage{listings}
\usepackage{inputenc}
\renewcommand{\lstlistingname}{Code}
\renewcommand{\lstlistlistingname}{\lstlistingname -Verzeichnis}

% Farben
\usepackage[table]{xcolor}
\definecolor{keywordLila}{RGB}{127,0,85}
\definecolor{stringBlau}{RGB}{42,0,255}
\definecolor{integerOrange}{RGB}{255,192,0}
\definecolor{codeBg}{RGB}{242,242,242}
\definecolor{grau}{RGB}{100,100,100}

\definecolor{hellgelb}{rgb}{1,1,0.9}
\definecolor{colKeys}{rgb}{0.8,0,0.5}
\definecolor{colIdentifier}{rgb}{0.6,0,0.3}
\definecolor{colComments}{rgb}{0,0.5,0}
\definecolor{colString}{rgb}{0,0,1}

% --------------------------------------------------------------------------------------------
% 
% Eigene Definitionen für Quelltext-Stile
%
% --------------------------------------------------------------------------------------------

%Definition von JCL
\lstdefinelanguage{JCL}{
keywords={SYSPRINT, SYSIN, EXEC, DD, PGM, SYSOUT, SET, MAXCC, JOB, MSGCLASS, CLASS, MSGLEVEL, NOTIFY, SYSUID, DISP, SHR, DSN, DB2XQL9, TRENNER, HEADER, P,  'XQL.SYSIN', 'XQL.SYSPRINT', NEW, CATLG, CATLG, UNIT, WORK, SPACE, TRK, RLSE, DCB, DSORG, PS},
keywordstyle=\color{keywordLila},
sensitive=false,
comment=[l]{//*},
commentstyle=\color{grau},
stringstyle=\color{stringBlau},
morestring=[b]',
morestring=[b]"
}

% SET  für den SQL-Code
\lstset{
language={SQL},
morekeywords ={USE, ADD, ENFORCED, REFERENCES},
frame={single},
captionpos={b},
keywordstyle={\color{keywordLila}},
stringstyle={\color{stringBlau}},
backgroundcolor={\color{codeBg}}
commentstyle=\color{black},
}

% language define für JSON
\lstdefinelanguage{json}{
string=[s]{"}{"},
comment=[l]{:},
frame={single},
captionpos={b},
keywordstyle={\color{keywordLila}},
stringstyle={\color{stringBlau}},
backgroundcolor={\color{codeBg}},
commentstyle=\color{red},
literate=
  {[}{{\textcolor{black}{[}}}{1}
  {]}{{\textcolor{black}{]}}}{1}
  {\{}{{\textcolor{keywordLila}{\{}}}{1}
  {\}}{{\textcolor{keywordLila}{\}}}}{1}
}

% PDF-Optionen -------------------------------------------------------------------------
\usepackage[
    bookmarks,
    bookmarksopen=true,
    colorlinks=true,
% diese Farbdefinitionen zeichnen Links im PDF farblich aus
    linkcolor=red, % einfache interne Verknüpfungen
    anchorcolor=black,% Ankertext
    citecolor=blue, % Verweise auf Literaturverzeichniseinträge im Text
    filecolor=magenta, % Verknüpfungen, die lokale Dateien öffnen
    menucolor=red, % Acrobat-Menüpunkte
    urlcolor=keywordLila,
% diese Farbdefinitionen sollten für den Druck verwendet werden (alles schwarz):
    % linkcolor=black, % einfache interne Verknüpfungen
    % anchorcolor=black, % Ankertext
    % citecolor=black, % Verweise auf Literaturverzeichniseinträge im Text
    % filecolor=black, % Verknüpfungen, die lokale Dateien öffnen
    % menucolor=black, % Acrobat-Menüpunkte
    % urlcolor=black, 
%    backref, % muss auf Grund von biblatex deaktiviert werden
    plainpages=false, % zur korrekten Erstellung der Bookmarks
    pdfpagelabels, % zur korrekten Erstellung der Bookmarks
%    hypertexnames=false, % zur korrekten Erstellung der Bookmarks, wegen biblatex deaktiviert
    linktocpage % Seitenzahlen anstatt Text im Inhaltsverzeichnis verlinken
]{hyperref}

% Befehle, die Umlaute ausgeben, führen zu Fehlern, wenn sie hyperref als Optionen übergeben werden
\hypersetup{
    pdftitle={\titel \untertitel},
    pdfauthor={\autor},
    pdfsubject={\titel \untertitel},
    pdfkeywords={\keywords},
}

% Hycap behebt hyperref Fehler
%\usepackage[all]{hypcap}


% fortlaufendes Durchnummerieren der Fußnoten -------------------------------
\usepackage{chngcntr}


% Caption
\usepackage{caption}

% Automatisches Abkürzungsverzeichnis
%\usepackage[printonlyused]{acronym}
\usepackage[nolist]{acronym}

% Für Enumerationes mit Buchstaben
\usepackage{enumerate}

% schönere Tabellen --------------------------------------------------------------------
\usepackage{tabularx}
\usepackage{multirow}

% für Fußnoten in Tabellen
%\usepackage{parnotes}

%Beispieltexte
\usepackage{blindtext}

% Für H in Bildern
\usepackage{float}
% Für. einstellbare seperatoren
\usepackage{booktabs}
